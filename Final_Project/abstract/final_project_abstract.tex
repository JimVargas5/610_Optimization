%Jim Vargas

\documentclass[12]{article}

\usepackage[margin=1in]{geometry}
\usepackage{setspace}
\usepackage{graphicx}
\usepackage{verbatim}
\usepackage{amsmath, amsfonts, amssymb}
\usepackage{dsfont}
\usepackage[none]{hyphenat}
\usepackage{float}

\begin{document}
\begin{center}
\textbf{Final Project Abstract}\\
MTH 610\\
Jim Vargas\\
\end{center}

The aim of this project is to compare two methods of obtaining solutions to Lorenz '63 model differential equation system
\begin{align*}
x'(t) &= \sigma (y-x) \\
y'(t) &= \rho x - y -xz \\
z'(t) &= xy - \beta z,
\end{align*}
where I will denote $\mathbf{X}(t) =[x(t),y(t),z(t)]^\top$ to represent a solution to the system compactly. I will start by computing a numerical solution using a 4th order Runge-Kutta Method (RK4) with 10,000 time steps. Afterwards, I will treat the RK4 solution as the "true state" and sample points of this solution at every tenth time step, with which I will use as 'observational data' to compute another numerical solution to the Lorenz model using an extended Kalman filter (EKF) on a sparser time scale (1,000 time steps). I will use the EKF method three times, changing how I create observations. These observations will be taken as
\begin{align*}
\mathbf{y}=\mathbf{Hx}+\varepsilon_0
\end{align*}
where $\mathbf{y}\in \mathds{R}^m$, $m=3,2,1$ in cases 1, 2, and 3 respectively, and $\varepsilon_0 \in \mathds{R}^m$ is a random vector of observational errors. In the first case ($m=3$), $\mathbf{H} \in \mathds{R}^{3\times 3}$ will observe all three components of $\mathbf{X}$; in the second, $\mathbf{H}\in \mathds{R}^{2\times3}$ will observe the first two only; in the third, $\mathbf{H}\in \mathds{R}^{1\times 3}$ will observe the first only. Finally, I will compare each of these three EKF solutions to the RK4 solution of the Lorenz model.


\end{document}


